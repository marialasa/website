\documentclass[12pt]{article}
\usepackage[margin=1in]{geometry}
\usepackage[spanish]{babel}
\setlength{\parindent}{0pt}
\setlength{\parskip}{5pt}
\pagenumbering{gobble}

\usepackage{amsmath,amsthm,amssymb}
\usepackage{graphicx}
\usepackage{float}
\usepackage[utf8x]{inputenc}
\usepackage{lmodern,textcomp}
\usepackage{wasysym}
\usepackage{hyperref}
\usepackage[dvipsnames]{xcolor}
\usepackage{hyperref}
\hypersetup{
    colorlinks=true,
    linkcolor=blue,
    filecolor=blue,      
    urlcolor=blue,
}

\begin{document}

\title{Curriculum Vit\ae}
\author{\bf María de los Ángeles Lasa \\ \href{mailto:ma.angeleslasa@yahoo.com}{ma.angeleslasa@yahoo.com}}
\date{}

\maketitle

\section*{Estudios}

\begin{rSection}

{\bf University of Oxford}, St Antony's College \hfill {\em 2018-2019} \\ 
Master of Public Policy, Second-Class Honours (2:1). Trabajo final: Merit.\\

{\bf Università degli Studi di Camerino}, School of Advanced Studies \hfill {\em 2011-2014} \\ 
Ph.D. in Political Science, Insigni Cum Laude (Ottimo). \\

{\bf Universidad Católica de Córdoba}, Facultad de Ca. Política y RR.II. \hfill {\em 2005-2009} \\ 
Licenciatura en Relaciones Internacionales, 9,05/10.

\end{rSection}

\section*{Otros estudios de posgrado}

\begin{itemize}

    \item {\bf Universidad Torcuato Di Tella} (AR) \hfill {\em 2020} \\ 
    Programa Ejecutivo en Ciencia de Datos para Políticas Públicas

    \item {\bf Tel Aviv University} (IL) \hfill {\em 2019} \\ 
    Oxford-Blavatnik Policy Programme

    \item {\bf Universitat Pompeu Fabra} (SP) \hfill {\em 2017} \\ 
    Posgrado en Diseño y Evaluación de Políticas Públicas
    
    \item {\bf Universidad de San Andrés} (AR) \hfill {\em 2016} \\ 
    Programa en Gobernabilidad, Gerencia Política y Gestión Pública

    \item {\bf University of Massachusetts Amherst} (US) \hfill {\em 2014} \\ 
    Civic Initiative's Argentinian Fulbright Young Leaders Program
    
    \item {\bf Universidad Complutense de Madrid} (SP) \hfill {\em 2011} \\ 
    Escuela de Verano en Diplomacia y Medios de Comunicación

\end{itemize}

\section*{Docencia e investigación}

\begin{rSection}

{\bf University of Oxford} (UK) \hfill {\em 2020-Continúa} \\ 
Data Analyst en el proyecto \href{https://www.bsg.ox.ac.uk/research/research-projects/oxford-covid-19-government-response-tracker}{\em Oxford COVID-19 Government Response Tracker (OxCGRT)}, Blavatnik School of Government. Director: Thomas Hale, Ph.D.\\

{\bf Universidad Nacional de Villa María} (AR) \hfill {\em 2018-2019} \\ 
Investigadora asistente en el proyecto {\em Interioridad y providencia en el Theophrastus Redivivus}. Programa de Incentivos a la Investigación del Ministerio de Educación de la República Argentina. Director: Dr. Carlos Daniel Lasa.\\

{\bf Universidad Católica de Córdoba} (AR) \hfill {\em 2018-Continúa} \\ 
Evaluadora en {\em Crítica y resistencias. Revista de Conflictos Sociales Latinoamericanos}, Facultad de Ciencia Política y Relaciones Internacionales. ISSN: 2525-0841.\\

{\bf Universidad de Buenos Aires} (AR) \hfill {\em 2016} \\ 
Profesora invitada, Seminario de Geopolítica, Facultad de Ciencias Sociales. Prof. Titular: Dr. Julio Burdman.\\

{\bf Universidad Católica de Córdoba} (AR) \hfill {\em 2015} \\ 
Becaria de investigación en el proyecto {\em Cooperación técnica de la Provincia de Córdoba (2010-2015). Historia y perspectivas}, Facultad de Ca. Política y Relaciones Internacionales. \\

{\bf Universidad de Los Andes} (CO) \hfill {\em 2013} \\ 
Investigadora visitante, Departamento de Ciencia Política. Tutora: Arlene Tickner, Ph.D. \\

{\bf University of Texas at Austin} (US) \hfill {\em 2013} \\ 
Visiting Researcher, Lozano Long Institute of Latin American Studies (LLILAS). Tutor: Charles R. Hale, Ph.D. \\

{\bf Universidad Nacional de Villa María} (AR) \hfill {\em 2009} \\ 
Adscripta, Cátedra de Introducción a las Relaciones Internacionales, Facultad de Ciencias Sociales. Prof. Titular: Mg. Gustavo Luque.\\

{\bf Universidad Católica de Córdoba} (AR) \hfill {\em 2008-2009} \\ 
Asistente de investigación, Observatorio de Valores en la Sociedad Internacional, Facultad de Ciencia Política y Relaciones Internacionales. Prof. Titular: Mg. Nelson Specchia.\\

{\bf Universidad Católica de Córdoba} (AR) \hfill {\em 2006-2008} \\ 
Ayudante alumna, Cátedra de Historia de las Relaciones Internacionales, Facultad de Ciencia Política y Relaciones Internacionales. Prof. Titular: Mg. Viviana Arias.\\

{\bf Universidad Católica de Córdoba} (AR) \hfill {\em 2006} \\ 
Ayudante alumna, Cátedra de Metodología I, Facultad de Ciencia Política y Relaciones Internacionales. Prof. Titular: Dra. Alejandra Ciuffolini.

\end{rSection}

\section*{Experiencia profesional}

\begin{rSection}

{\bf Movilizatorio} (CO) \hfill {\em 2020-Continúa} \\ 
Gerenta de Alianzas y Proyectos para Argentina · {\em Social lab} de Purpose.com para América Latina.\\

{\bf Organisation for Economic Co-operation and Development} (FR) \hfill {\em 2019} \\ 
Research \& Advice · Open Government Unit, Departamento de Policy Evaluation \& Public Governance Reviews, Directorate for Public Governance.\\

{\bf Buenos Aires 2018 Youth Olympic Games} (AR) \hfill {\em 2017-2018} \\ 
Gerenta de Sostenibilidad \& Cooperación Internacional · Dirección General de Legado \& Sostenibilidad, Departamento de Corporate Planning.\\

{\bf Unidad de Información Financiera} (AR) \hfill {\em 2016-2017} \\ 
Técnica Experta en Cooperación Internacional AML/CFT · Departamento de Asuntos Internacionales, Ministerio de Finanzas de la República Argentina.\\

{\bf Gobierno de la Ciudad de Buenos Aires} (AR) \hfill {\em 2015-2016} \\ 
Joven Profesional · Programa Jóvenes Profesionales, Jefatura de Gabinete de Ministros.\\

{\bf Municipalidad de Villa María} (AR) \hfill {\em 2010-2011} \\ 
Asistente Profesional Jr. · Departamento de la Función Pública.

\end{rSection}

\section*{Publicaciones}

\begin{rSection}

{\bf Artículos Peer-Reviewed}

\end{rSection}

\begin{enumerate}

\item  LASA, C. D. \& LASA, M. (2018). “El Theophrastus Redivivus. Religión y poder político a mediados del Siglo XVII”. {\it Studia Politic\ae}, 14 (41): p. 137-158.

\item  LASA, M. (2017). “Cooperación técnica internacional de la Provincia de Córdoba (2005-2015). Historia y perspectivas”. {\it Breviario en Relaciones Internacionales de la Universidad Nacional de Córdoba}, 35 (2): p. 34-57.

\item  RICCI, G. \& LASA, M. (2015). “Hacia una definición de complejo coca-cocaína”. {\it Studia Politic\ae}, 12 (31): p. 75-97.

\item  LASA, M. (2013). “Challenges, Cooperation and Paradoxes in the Coca-Cocaine Complex”. {\it Politikon: the IAPSS Journal of Political Science}, 5 (19): p. 3-14.

\end{enumerate}

\begin{rSection}

{\bf Capítulos de libros}

\end{rSection}

\begin{enumerate}

\item  LASA, M. (2013). “Coca, cocaína y lucha contra el narcotráfico en Bolivia”. En SPECCHIA, Nelson \& CAMPS, Hernán (Eds.) Bolivia. {\it La refundación multiétnica sobre la riqueza del Potosí}, Córdoba: EDUCC, p. 225-244.

\item  COSTANTINO, S. \& LASA, M. (2010). “Las industrias de las FARC: secuestro y narcotráfico”. En SPECCHIA, Nelson (Ed.). {\it El último año de las FARC}, Córdoba: EDUCC, p. 79-90. 

\item  COSTANTINO, S.; GIAMPIERI, A. \& LASA, M. (2008). “La sofisticación de la [exclusión] en los barrios-ciudades de Córdoba, los nuevos ghettos del siglo XXI”. En KOLEFF, Miguel (Ed.) {\it Acerca del Reconocimiento del otro en la cultura contemporánea}, Córdoba: EDUCC, p. 21-34.

\end{enumerate}

\begin{rSection}

{\bf Citas en libros y tesis doctorales}

\end{rSection}

\begin{enumerate}

\item  AUYERO, J. \& SOBERING, K. (2019). {\it The Ambivalent State. Police-Criminal Collusion at the Urban Margins}. Oxford: Oxford University Press, p. 45-46.

\item  PARK, H. (2018). {\it North Korean Migrants in China: A Case Study of Human Smuggling and Trafficking}. Rutgers University: Ph.D. Diss., p. 72.

\item  SAIN, M. (2017). {\it ¿Por qué preferimos no ver la inseguridad? (aunque digamos lo contrario)}. Buenos Aires: Siglo Veintiuno Editores.

\end{enumerate}

\begin{rSection}

{\bf Otras publicaciones}

\end{rSection}

\begin{enumerate}

\item ARGARAÑÁZ, M.; LASA, M. {\it et al.} (2018). \href{https://www.ar.undp.org/content/argentina/es/home/library/environment_energy/PortocoloES.html}{\it Protocolo para la gestión de eventos sostenibles}, Buenos Aires: Naciones Unidas Argentina \& UNDP.

\item BER, M.; DHRU, K. \& LASA, M. (2011). “Rethinking leadership, re-founding leadership”. En {\it St. Gallen Wings of Excellence Award}, publicación del St. Gallen Symposium. St. Gallen: Universität St. Gallen, p. 24-31.

\item DE MATTEI, Roberto (2007). {\it La soberanía necesaria. Reflexiones sobre la crisis del Estado Moderno}, Guadalajara: Folia Universitaria. Traducción al español, notas y prólogo a cargo de LASA, M.\\ \\

\end{enumerate}

\begin{rSection}

{\bf Artículos de prensa (selección)}

\end{rSection}

\begin{enumerate}

\item \href{http://www.laprensa.com.ar/484388-La-odisea-argentina.note.aspx}{\it La Odisea argentina}. Sección Política, 30.12.2019. La Prensa.

\item \href{https://www.perfil.com/noticias/opinion/el-efecto-potemkin.phtml}{\it El efecto Potemkin}. Sección Opinión \& Análisis, 18.08.2019. Perfil.com

\item \href{https://blogs.bsg.ox.ac.uk/2019/05/30/fernandez-fernandez-centrist-credibility-in-the-argentina-election/}{\it Fernández-Fernández: centrist credibility in the Argentina election?} Sección Governance \& Law, 30.05.2019. BSG Blog, University of Oxford.

\item \href{https://blogs.bsg.ox.ac.uk/2018/10/31/the-chameleonic-paradox-of-argentinas-foreign-policy/}{\it The chameleonic paradox of Argentina’s foreign policy}. Sección Governance \& Law, 31.10.2018. BSG Blog, University of Oxford.

\item \href{https://www.perfil.com/noticias/internacional/9-de-noviembre-del-muro-de-berlin-al-eeuu-de-trump.phtml}{\it 9 de noviembre: del Muro de Berlín a Donald Trump}. Sección Internacional, 9.11.2016. Perfil.com.

\item \href{https://www.lanacion.com.ar/opinion/narco-made-in-argentina-nid1834560}{\it Narco made in Argentina}. Sección Ideas, 10.07.2015. La Nación.

\item \href{https://www.perfil.com/noticias/politica/el-caso-schwarzenegger-el-impacto-del-sexo-y-el-escandalo-en-la-arena-politica-20110529-0029.phtml}{\it El caso Schwarzenegger: el impacto del sexo y el escándalo en la arena política}. Sección Internacional, 29.05.2011. Perfil.com.

\item \href{https://www.perfil.com/noticias/internacional/espana-que-es-lo-que-reclamas-20110522-0005.phtml}{\it España, ¿qué es lo que reclamas?} Sección Internacional, 22.05.2011. Perfil.com.

\end{enumerate}

\section*{Exposiciones}

\begin{enumerate}

\item Sesión académica virtual \href{https://www.facebook.com/watch/live/?v=2955083387939317}{\it América Latina y el Covid-19: una perspectiva desde las políticas públicas}, Universidad Comunera. Asunción (PY), 9 de junio de 2020. \underline {Ponencia}: “Diseño de políticas públicas en contextos de incertidumbre radical”.

\item Ciclo Anual de Conferencias de la Biblioteca Dr. Alberto Caturelli, Sociedad Educativa Argentina. Villa María (AR), 21 de noviembre de 2019. \underline {Ponencia}: “Urgentina: federalismo y buropatías de un país pulpo”.

\item Ciclo Anual de Conferencias de la Biblioteca Dr. Alberto Caturelli, Sociedad Educativa Argentina. Villa María (AR), 26 de noviembre de 2018. \underline {Ponencia}: “Geopolítica, historia y diplomacia del Reino Hermético. Corea del Norte, 1948-2018”.

\item {\it XIII Congreso Nacional de Ciencia Política: La política en entredicho. Volatilidad global, desigualdades persistentes y gobernabilidad democrática}, SAAP \& Universidad Torcuato Di Tella. Buenos Aires (AR), 2-5 de agosto de 2017. \underline {Ponencia}: “Megaeventos y desarrollo local: los Juegos Olímpicos y sus estrategias de construcción de legado”. 

\item \href{https://www.youtube.com/watch?v=DnXO4vuB9bU}{TEDxRecoleta}, Biblioteca Nacional de Buenos Aires. Buenos Aires (AR), 9 de octubre de 2016. \underline {Presentación}: “La década becada”.

\item \href{https://www.youtube.com/watch?v=aUp86rzTjew}{TEDxUniversidadCatólicadeCórdoba}, Universidad Católica de Córdoba. Córdoba (AR), 2 de septiembre de 2016. \underline {Presentación}: “Ajedrez global: la coronación del peón”.

\item {\it VII Congreso Latinoamericano de Ciencia Política}, Asociación Latinoamericana de Ciencia Política (ALACIP) \& Universidad de Los Andes. Bogotá (CO), 25-27 de septiembre de 2013. \underline {Ponencia}: “La seguridización del complejo coca-cocaína en la Argentina de la post-crisis 2001”.

\item {\it II Convegno Internazionale di Studi di Droga \& Società: Criminalità organizzata e intervento sociale}, Osservatorio Nazionale di Abusi Psicologici \& Universidad de San Buenaventura. Roma (IT), 12 de mayo de 2012. \underline {Ponencia}: “Il consumo della droga di scarto nell’Argentina post-crisi. Anche l’Italia in pericolo?”

\item {\it Il volto della terra: Ambiente ed Illegalità}, Università degli Studi di Camerino \& Società Italiana di Geologia Ambientale. Camerino (IT), 19 de mayo de 2011. \underline {Ponencia}: “Coca, cocaina e criminalità in Argentina (2001-2010)”.

\item {\it 41º St. Gallen Symposium: Just Power}, Universität St. Gallen. St. Gallen (CH), 9-13 de mayo de 2011. \underline {Ponencia}: “Re-thinking leadership, re-founding leadership”.

\item {\it Strengthening Responsible Business and Governance in Africa and Latin America}, World Bank Institute. Bruselas (BE), 16-18 de noviembre de 2010. \underline {Ponencia}: “(Hacer) + (No hacer): un binomio contra la corrupción”.

\item {\it World Bank ABCDE 2010: Development Challenges in a Post-Crisis World}, World Bank Group. Estocolmo (SE), 31 de mayo – 2 de junio de 2010. \underline {Ponencia}: “Todo Jóvenes: From a Near-sighted Princess to a Change in Focus”. 

\item {\it II Jornadas de Ciencia Política: los nuevos procesos de democratización en Argentina y América Latina}, Facultad de Ciencias Sociales, Universidad Nacional de Villa María. Villa María (AR), 14-15 de mayo de 2009. \underline {Ponencia}: “Narcotráfico y fallas telúricas en las democracias latinoamericanas contemporáneas”.

\item {\it IV Parlamento Universitario Latinoamericano}, Universidad Católica Argentina. Pilar (AR), 26-28 de agosto de 2008. \underline {Ponencia}: “Amenazas a la soberanía estatal: los grupos insurgentes y su relación con el crimen transnacional”.

\item {\it VI Jornadas Interdisciplinarias: Acerca del Reconocimiento del Otro en la Cultura Contemporánea}, Facultad de Filosofía y Humanidades, Universidad Católica de Córdoba. Córdoba (AR), 6-7 de junio de 2008. \underline {Ponencia}: “La sofisticación de la [exclusión] en los barrios-ciudades de Córdoba, los nuevos {\it ghettos} del siglo XXI”.

\end{enumerate}

\section*{Becas}

\begin{enumerate}

\item Subsidio de alojamiento de la {\bf Casa Argentina en París}, dependiente del Ministerio de Educación de la República Argentina, para summer placement en la Organización para la Cooperación y el Desarrollo Económicos (OECD). París (FR), 2019.

\item Beca {\it Summer Internship Programme 2019} de {\bf Erasmus+} para summer placement en la Organización para la Cooperación y el Desarrollo Económicos (OECD). París (FR), 2019. Dotación económica: \textcolor{red}{€800}.

\item Beca de {\bf iTrek} para participar del {\it Oxford-Blavatnik Policy Programme}. Tel Aviv (IL), 2019.

\item Beca de la {\bf Blavatnik School of Government} para estudios de maestría en el Reino Unido. Dotación económica (2018-2019): \textcolor{red}{£7.376}.

\item Beca {\bf Chevening} para estudios de maestría en el Reino Unido. Dotación económica (2018-2019): \textcolor{red}{£47.569}.

\item Beca parcial del {\bf Instituto de Investigación y Educación Económica (I+E)} para cursar el {\it Posgrado en Diseño y Evaluación de Políticas Públicas}, Universitat Pompeu Fabra. Buenos Aires (AR), 2017.

\item Beca parcial de la {\bf Universidad de San Andrés} para cursar el {\it Programa de Posgrado en Gobernabilidad, Gerencia Política y Gestión Pública}, Universidad de San Andrés, The George Washington University \& Banco de Desarrollo de América Latina (CAF). Buenos Aires (AR), 2016.

\item Beca de la {\bf Konrad-Adenauer-Stiftung (KAS)} para cursar el {\it Diplomado en Transparencia Institucional}, KAS \& Legislatura Porteña. Buenos Aires (AR), 2016. Dotación económica: \textcolor{red}{U\$270}.

\item Beca del {\bf Centro para la Apertura y el Desarrollo de América Latina (CADAL)} para cursar el {\it Seminario sobre las transiciones comunistas en Europa del Este}, CADAL. Buenos Aires (AR), 2016.

\item Beca de la {\bf Red Argentina de Centros de Estudios Internacionales (RACEI)} para cursar el {\it Seminario de Posgrado sobre la Política Exterior de Estados Unidos hacia América Latina}, Universidad de Buenos Aires \& Comisión Fulbright. Buenos Aires (AR), 2015. Dotación económica: \textcolor{red}{U\$150}.

\item Beca de Investigación de la Secretaría de Integración y Relaciones Internacionales del {\bf Gobierno de la Provincia de Córdoba}. Córdoba (AR), 2015. Dotación económica: \textcolor{red}{U\$2.500}.

\item Beca completa + subsidio de viaje y alojamiento de la {\bf Embajada de los Estados Unidos de América} para cursar la {\it Escuela de Verano sobre Política de Estados Unidos}, Universidad de San Andrés. Victoria (AR), 2015.

\item Beca completa + subsidio de viaje y alojamiento de {\bf Comisión Fulbright} para cursar el {\it Civic Initiative’s Argentinian Fulbright Young Leaders Program}, University of Massachusetts Amherst. Amherst (US), 2014. Dotación económica: \textcolor{red}{U\$8.000}.

\item Beca completa + subsidio de viaje de {\bf Comisión Fulbright} para cursar el {\it Seminario para Jóvenes Líderes sobre Estados Unidos}, Universidad de San Andrés. Buenos Aires (AR), 2013.

\item Subsidio de viaje y alojamiento de la {\bf Universität St.Gallen} para asistir al {\it 43º St. Gallen Symposium: Rewarding Courage}. St. Gallen (CH), 2013.

\item Subsidio de viaje y alojamiento del {\bf Banco Mundial} para asistir al {\it Young Professionals Summit}, German Marshall Fund of the United States. Bruselas (BE), 2013.

\item Subsidio de viaje y alojamiento de {\bf European Alternatives} para asistir al Foro {\it Legality and Struggle against Mafias in Europe. A transnational response to the transnational phenomenon of organized crime}, European Alternatives \& FLARE. Sofía (BG), 2011. Dotación económica: \textcolor{red}{€350}.

\item Beca del {\bf Consorzio Interuniversitario Italiano per l’Argentina (CUIA)} para estudios de doctorado en Italia. Dotación económica (2011-2014): \textcolor{red}{€48.476,58} + exención de matrícula.

\item Beca + subsidio de viaje y alojamiento de {\bf Fundación Carolina} \& {\bf Banco Santander} para cursar la {\it Escuela de Verano en Diplomacia y Medios de Comunicación}, Universidad Complutense de Madrid. Madrid (SP), 2011. Dotación económica: \textcolor{red}{€4.000}.

\item Subsidio de viaje y alojamiento de la {\bf Universität St.Gallen} para asistir al {\it 41º St. Gallen Symposium: Just Power}, Universität St.Gallen. St. Gallen (CH), 2011.

\item Subsidio de viaje y alojamiento del {\bf Banco Mundial} para asistir a las siguientes conferencias: {\it Strengthening Responsible Business and Governance in Africa} \& {\it International Leadership Symposium}, Parlamento Europeo. Bruselas (BE), 2010.

\item Subsidio de viaje y alojamiento del {\bf Banco Mundial} para asistir a la conferencia {\it ABCDE 2010: Development Challenges in a Post-Crisis World}, Banco Mundial. Estocolmo (SE), 2010.

\item Subsidio de alojamiento del {\bf Instituto Tecnológico de Buenos Aires (ITBA)} para asistir al {\it South American Business Forum: New Paradigms, New Challenges}, ITBA. Buenos Aires (AR), 2009.

\item Beca de {\bf Fundación Carolina} para asistir al {\it Seminario Internacional de Posgrado “Viviendo España: una mirada sociopolítica a la historia y el presente de España”}, Fundación Carolina \& Consejo Argentino para las Relaciones Internacionales (CARI). Buenos Aires (AR), 2009.

\item Beca de {\bf Comisión Fulbright} para asistir al {\it Seminario Internacional “The United States today: a Challenge for the Constitutional System in the Twenty-First Century”}, Universidad Nacional de Córdoba. Córdoba (AR), 2009.

\item Beca de la {\bf Universidad Católica de Córdoba} para asistir al {\it IV Parlamento Universitario Latinoamericano}, Universidad Católica Argentina. Pilar (AR), 2008. Dotación económica: \textcolor{red}{U\$350}.

\end{enumerate}

\section*{Premios \& distinciones}

\begin{enumerate}

\item {\bf Mención de Honor} al proyecto {\it Juegos Olímpicos de la Juventud y desarrollo local: una propuesta de gestión bottom-up del legado tangible de Buenos Aires 2018}. Juegos Olímpicos de la Juventud Buenos Aires 2018, Ernst & Young (EY) \& Banco Interamericano de Desarrollo (BID), 2017. 

\item {\bf Mención de Honor} del Programa en Gobernabilidad, Gerencia Política y Gestión Pública al proyecto {\it Villa Olímpica de la Juventud: integración social desde un enfoque sustentable}. Universidad de San Andrés, George Washington University \& Banco de Desarrollo de América Latina (CAF), 2016. 

\item {\bf Joven Sobresaliente} de la Provincia de Córdoba. Cámara de Comercio de Córdoba, 2015.

\item {\bf Premio a la Excelencia en Investigación}. Università degli Studi di Camerino (IT), 2013. Premio: \textcolor{red}{€3.025,87}.

\item {\bf Leader of Tomorrow}. St. Gallen Symposium \& Universität St.Gallen (CH), 2013. Seleccionada para participar de workshop sobre tendencias económicas globales dictado por Christine Lagarde.

\item {\bf Young Professional Leader}. German Marshall Fund of the United States, 2013. Candidatura auspiciada por el Banco Mundial. 

\item {\bf St. Gallen Wings of Excellence Award}: 3º Premio. St. Gallen Symposium \& Universität St.Gallen (CH), 2011. Premio: \textcolor{red}{€4.000}.

\item {\bf Premio Ciudadana Destacada}. Concejo Deliberante de Villa Nueva, 2010.

\item {\bf World Bank Ideas 4 Action Competition}: 1º Premio. Instituto del Banco Mundial \& Gobierno de Bélgica, 2010. Premio: \textcolor{red}{€500}.

\item {\bf Essay Contest for Young People}: Mención de Honor. Goi Peace Foundation (JP) \& UNESCO, 2010. 

\item {\bf World Bank International Essay Competition}: 2º Premio. Banco Mundial \& Gobierno de Suecia, 2010. Premio: \textcolor{red}{U\$1.000}. El Poder Legislativo de la Provincia de Córdoba (AR), expresó su beneplácito por la obtención de este premio en su \href{http://datos.legiscba.gob.ar/contenidos/themes/Legislatura-th01/descarga_documento.php}{declaración D-11149/10 (Período 132)}, del 28 de julio de 2010.

\item {\bf Competencia de artículos para estudiantes y graduados}: 1º Premio, Facultad de Filosofía y Humanidades, Universidad Católica de Córdoba (AR), 2008.

\end{enumerate}

\section*{Afiliación institucional}

\begin{itemize}

\item {\bf Nacionales}: Chevening Alumni Argentina, Consejo Argentino para las Relaciones Internacionales (CARI), Sociedad Argentina de Análisis Político (SAAP).

\item {\bf Internacionales}: Oxford Union Society, R-Ladies Global, Young Leaders of the Americas Initiative.

\end{itemize}

\section*{Voluntariado}

\begin{itemize}

\item {\bf Oxford Development Consultancy (ODC)} \hfill {\em 2018-2019} \\
Consultora {\it pro bono} para bancos de micro-créditos de El Salvador
    
\item {\bf Asociación Puentes Enteros} \hfill {\em 2015-2018} \\
Asesora, capacitadora y jurado de Modelos de Naciones Unidas (MNU)
    
\item {\bf South American Business Forum} \hfill {\em 2013-Continúa} \\
Jurado del Concurso de Ensayos
    
\item {\bf Todo Jóvenes} \hfill {\em 2009-2013} \\
Fundadora \& Directora
  
\end{itemize}
\section*{Informática}
\begin{rSection}
    {\bf Software}: MS Excel, RStudio, SPSS. {\bf Lenguajes}: R, HTML, \LaTeX, CSS.
    
\end{rSection}

\section*{Idiomas}
\begin{rSection}
    {\bf Castellano}: nativo. {\bf Inglés}: avanzado (IELTS 7.5). {\bf Italiano}: intermedio (B2).
    
\end{rSection}

\vfill

\date{Actualizado al \today}
\end{document}
